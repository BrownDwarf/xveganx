%\documentclass[iop,revtex4]{emulateapj}% change onecolumn to iop for fancy, iop to twocolumn for manuscript
\documentclass[twocolumn]{emulateapj}% change onecolumn to iop for fancy, iop to onecolumn for manuscript
%\documentclass[12pt,preprint]{aastex}

%\usepackage{lineno}
%\usepackage{blindtext}
%\linenumbers
%\usepackage{breqn}
\usepackage{amsmath}

\let\pwiflocal=\iffalse \let\pwifjournal=\iffalse
%From: http://arxiv.org/format/1512.00483
%\input{setup}
\input{mgs_setup}


\newcommand{\iancze}{{\sc C15}}

\providecommand{\eprint}[1]{\href{http://arxiv.org/abs/#1}{#1}}
\providecommand{\adsurl}[1]{\href{#1}{ADS}}
\newcommand{\name}{LkCa 4 }
\newcommand{\project}[1]{\textsl{#1}}
%\def\vsini{$v\sin{i_*}$}


\slugcomment{In preparation}

\shorttitle{Extreme Starspots in Taurus}

\shortauthors{Gully-Santiago et al.}

\bibliographystyle{yahapj}

\begin{document}

\title{The coverage fraction of starspots on young Taurus members and consequences for stellar ages estimates at 1$-$10 Myr}

\author{Michael A. Gully-Santiago,\altaffilmark{1}, author list TBD}


\altaffiltext{1}{Kepler/K2 Guest Observer Office, NASA Ames Research Center, Moffett Field, CA, USA}

\begin{abstract}

We assess starspot temperatures and coverage fraction for three heavily spotted Taurus pre-main sequence young stellar objects from their high-resolution, high bandwidth near infrared IGRINS spectroscopy.  We forward model the spectra with a mixture of cool and hot photosphere components using the \texttt{Starfish} spectral inference framework.  The disk-free, non-accreting sample members have 10$-$30 years worth of polychomatic photometry, which we use in four ways: first, to register the acquisition time of the spectroscopic measurement to the stellar rotational phase; second, to derive the secular changes of areal coverage fraction over 1 to 30 year timescales; third, to derive rotation periods; and fourth, to assign lower-limits to areal filling factors based on the amplitude of photometric variation and limiting case of non-emitting equatorial spots.  We derive starspot temperatures in the range of $x$ to $y$ K, and filling factors of $v$ to $z$.  These filling factors are typically $k$ to $m$ times higher than the limiting case of non-emitting equatorial spots, implying that either the stars possess large circumpolar starspots with the stellar spin axis inclined moderately towards our reference frame, or that the stars possess three or more dominant temperature components, rendering our two-temperature retrieval process flawed.  The filling factors and temperatures cause effective temperature estimates to systematically decrease, confounding age and mass estimates of these young stars and by extension all previous spectral-type-based estimates of fundamental properties of heavily spotted stars, though more work is needed.  The observations of secularly varying photometric amplitudes are consistent with the physical interpretations that some or all young stars pass through short-lived phases of enhanced starspot coverage, or that starspots migrate from near the poles towards the equator over time.

\end{abstract}

\keywords{stars: fundamental parameters ---  stars: low-mass --- stars: statistics}

\maketitle

\section{Introduction}\label{sec:intro}

The evolutionary state of a star can be superbly predicted with an age and a mass.  Angular momentum and metallicity further improve those predictions.  Unfortunately the age of a star cannot be trivially measured.  Pre-main sequence isochronal methods provide only coarse estimates of cluster ages, with perhaps 100\% uncertainties attributable to both intrinsic scatter and systematic bias.  Isochronal models generally ignore angular momentum, a safe bet after stars have resided on the pre-main-sequence for much longer than their formation timescale.  Pre-main sequence stars differ.  For the first $1-100$ Myr of a star's life, the rotation rate of a star may inhibit convective efficiency, making the star dwell in its puffed-up, larger-than-expected radius, in order to allow the same amount of internal energy to escape.  The extra-high rotation rates of young stars interplay with both global and surface magnetic fields and environmental factors to yield complex evolutionary phenomena.

One observational manifestation of rotationally-heightened magnetic fields is the presence of \emph{starspots}.  Starspots serve as a familiar---albeit inexact---avenue through which we can quantify possible limitations to isochronal age estimates.  Importantly, starspots possess many favorable observational properties that form the central thrust of this paper.  In our previous paper \citep{2017ApJ...836..200G}, we combined multiple datasets to reveal a large $\sim75\%$ coverage fraction of starspots on the surface of the Weak-Lined T-Tauri Star (WTTS) LkCa~4.  Here we seek to identify whether LkCa~4 merely represents an extreme outlier, or whether other WTTS's can possess similiarly large starspots.

%Fill in more thereoy here
\citet{somers15} emphasized the roll of starspots in inferring stellar ages.

We first emphasize possible geometric degeneracies that can give rise to larger-than-expected starspots.  We then collate numerous photometric and spectroscopic data sources (Section XXX), and their starspot-centric analyses (Section XXX).  We combine inferences on our small sample of stars to assess the effect of starspots on the pre-main sequence HR-diagram (Section XXX).  Finally we question our initial assumptions in Section XXX.

\section{Methods}\label{sec:methods}

\textbf{Our limited understanding of starspot properties, observational biases}
Observational limitations usually preclude a complete accounting of a star's surface structures.  In this section we consider a range of extreme limiting cases of starspot geometries as a guide to interpreting observations.  These admittedly pathological limiting cases do not represent physically plausible scenarios.  Instead they serve as benchmarks beyond which no further physical models can exist without changing assumptions or dismissing observations.

In this paper we will assume the target stars are all \emph{starspot dominated}, meaning that in the observations of interest, the dominant variance attributable to stellar activity arises from the presence of darker-than-average regions with characteristic temperatures cooler than their ambient photospheres, and not from brighter-and-hotter-than-average regions such as faculae or plages.  It is not yet known in which scenarios this assumption holds, but early evidence suggests that rapidly rotating sun-like stars are more likely to be starspot dominated than plage dominated in the visible waveband \citep{2017ApJ...851..116M}.  Tomographic maps of rapidly rotating young stars also indicate starspot domination \citep{donati14}.  While most results of the paper work equally well by swapping faculae-dominated for starspot-dominated, the assumptions utterly fail if we allow starspots and faculae to have comparable effect sizes on the same target star.

This paper focuses on methodologies applicable to disk-free, accretion-free rotating young stars, which serve as a great testbed for understanding starspots owing to their large surface magnetic fields and large amplitude of photometric variability.  The methods should also apply to main sequence and post-main sequence stars, albeit at typically diminished effect sizes.


\subsection{Limiting cases for interpreting monochromatic lightcurves}
The appearance and disappearance of starspots onto the unresolved hemisphere of a rotating star results in familiar rotationally-modulated stellar lightcurves.  Monochromatic lightcurves cannot be inverted unambiguously to yield a mapping of surface structures.  Young stellar lightcurves rarely show flat-topped lightcurves indicative of a single starspot.  Instead, most Kepler/K2 lightcurves of disk-free, accretion-free rotating young stars look like sine waves, or mixtures of sine waves.  The sine wave periods offer a simple interpretation-- the rotation period of the star if solid body rotation is assumed, or the rotation period evaluated at the latitude of the dominant starspot group in the case of differential rotation.  The sine wave amplitudes encode partial information about the starspot \emph{size} and temperature contrast to the ambient photosphere.  Unambiguously interpreting the absolute starspot \emph{size} and temperature from monochromatic lightcurve amplitudes requires additional geometric assumptions on the number, size, lifetimes, latitudinal and longitudinal distributions, and clustering properties of spots.  Observations of these quantities exists only for the sun, so their adoption lacks justification in anything but solar analogs.  Figure \ref{fig:cartoon} illustrates four examples of disparate stellar surface morphologies that possess comparable or identical photometric amplitudes and periods.

The first scenario illustrates a solitary non-emitting (i.e. zero Kelvin) starspot near the equator of a modestly inclined star.  Ignoring limb darkening, the peak-normalized, peak-to-valley amplitude of of the lightcurve is equal to the areal filling factor of the starspot.

The second scenario is pathological, but informative.  Consider an equator-on star in which one entire hemisphere (e.g. Eastern Hemisphere) possesses an isothermal photosphere, and the diametrically opposed hemisphere (e.g. Western Hemisphere) possess a cooler isothermal temperature.  The cooler hemisphere represents the lightcurve minimum, in which the observer receives less band-averaged flux simply because $T^4$ is smaller.  No known physical process is capable of delivering such a pathological structure, but nevertheless the temperature difference between the two hemispheres represents the minimum possible temperature contrast of a starspot, assuming the spots are isothermal.  The peak-normalized, peak-to-valley amplitude of of the lightcurve is equal to:


\begin{equation}
1- \int \frac{f_{\rm cool}(\lambda)}{f_{\rm ambient}(\lambda)} \epsilon(\lambda) d\lambda
\end{equation}

where $f_{\rm cool}$ and $f_{\rm ambient}$ refer to the flux of the photospheres for the cooler and hotter hemispheres respectively, and the integration includes the wavelength-dependent instrumental response.

The third illustration depicts another pathological geometry: the \emph{pole-on} star has a starspot that grows and shrinks cyclicly with a period $P \neq P_{\rm rot}$.  In this scenario, the rotation period is likely to be mis-identified.  The lightcurve amplitude depends on the size and temperature contrast of the spot.

The right-most illustration depicts a high-latitude active longitude in which the starspot is always visible from the observers perspective, but rotational modulation geometrically forshortens the perceived solid angle of the spot owing to projection effects, and limb darkening.  The amplitude of the lightcurve depends sensitively on the inclination of the star, the latitude of the spot, the wavelength-dependent limb darkening, the waveband of the observation, the temperature constrast of the spot, and the size of the spot.


\begin{figure*}
 \centering
 \includegraphics[width=0.98\textwidth, trim={0 5cm 0 4cm},clip]{figures/cartoon_illustration.pdf}
 \caption{Cartoon illustration of stars possessing different inclinations, starspot geometries, and positions, yet possessing the same lightcurve morphologies.}
 \label{fig:cartoon}
\end{figure*}

\subsection{Practical considerations for monochromatic lightcurves}

Deriving amplitudes from sinusoidal lightcurves should be as simple as measuring the peak-minus-valley flux differences.  In practice, additional factors complicate such a straightforward analysis.  One main uncertainty is how to treat \emph{secular} trends, in which the star's flux waxes and wanes with both an oscillatory function of periods $P_{\rm rot}$ and a slow-varying mean level with characteristic timescale $t_{\rm secular}$.  Secular changes in flux could be interpreted as a growing or waning average starspot coverage fraction as in a long-term stellar activity cycle.  The interpretation of secular trends depends on to-what-extent instrumental artifacts mimic these time-series structures.  In Kepler/K2, for example, spurious secular trends can exist at the few percent level as point spread functions leak out of pre-defined aperture masks due to spacecraft-induced image motion, thermal effects, and other instrumental systematic effects.  Increasing aperture size can sometimes, partially mitigate these biases at the expense of greater variance and read noise.  The characteristic timescales of these instrumental biases tend to resemble the observing season- either campaigns for K2, or quarters for Kepler prime.  Meanwhile, disk-free young stars exhibit bona fide secular trends of up to half a magnitude over periods of years-decades based on decades of ground-based photometric monitoring \citep{grankin08}.

In this paper we propose a practical compromise. We determine peak-normalized, peak-to-valley sinuoidal amplitudes at the rotation timescale from high-precision space-based time-series observations, conservatively deeming any secular trends as spurious.  We additionally determine a \emph{global} peak-normalized, peak-to-valley amplitude by analyzing archival ground based photometry spanning decades (\emph{c.f.} Section \ref{sec:archival_phot}).  This strategy resembles that adopted by \citet{2017ApJ...851..116M}, which interpreted variations in Kepler Full-Frame Images as activity cycles.

We measure sinusoidal amplitudes and spurious trends simultaneously with a phenomenological forward-modeling procedure that fits a sum of a sine wave and a polynomial of order $N_{\rm poly}$.  We determine the period $P$ of the sine wave from the multiterm Lomb-Scargle periodogram \citep{ivezic14}.  The linear model can then be solved analytically for the coefficients of the polynomial terms and the sine and cosine terms with linear least squares.  For clarity, fitting a straight line trend equates to a first order polynomial, $N_{\rm poly}=1$:


\begin{equation}
\hat f = c_0 + c_1 \cdot t + c_2 \cdot \sin{\frac{2\pi t}{P}} + c_3 \cdot \cos{\frac{2\pi t}{P}}
\end{equation}

A simple sine wave may not faithfully capture more complex oscillatory behavior, like piecewide flat-topped or sawtooth patterns.  We allow even more model flexibility by adding higher-order harmonic structures in a Fourier Series \citep{vanderplas15a} truncated to the first $N_{\rm Fourier}$ components:

\begin{equation}
  \begin{split}
 \hat f  & =  \sum_{i=0}^{N_{\rm poly}} c_i \cdot t^i \\
         & + \sum_{j=1}^{N_{\rm Fourier}} \Big( a_j \cdot \sin{\frac{2\pi j t}{P}} \\
         & + b_j \cdot \cos{\frac{2\pi j t}{P}} \Big)
\end{split}
\end{equation}

Our phenomenological forward model has a total of $N_{\rm poly}+2 \cdot N_{\rm Fourier}$ model parameters, representing the model complexity.  We set these numbers $N_{\rm poly}, N_{\rm Fourier}$ through a cross-validation procedure that balances the bias-variance tradeoff.  Once we have our model parameters in-hand, we overplot the observed lightcurve with a densely-sampled phenomenological model, examining the residuals for any additional noise structures, such as exoplanets or flares, which are masked and we iteratively re-compute the refined predicted flux time series.  We discard the nonzero ($i>0$) polynomial terms, yielding a constant plus harmonic model, which is then peak-normalized and deemed the trend-free stellar lightcurve $\hat f_{\rm \star}(t)$.  The detrended stellar amplitude is then: $$ A = 1 - \min{f_{\rm \star}(t)}$$

\subsection{Two-temperature spectral inference framework}

We employ the \texttt{Starfish}\footnote{Forked from the original (\href{https://github.com/iancze/Starfish}) at https://github.com/gully/Starfish} \citep{czekala15} spectral inference framework, as adapted from  by \citet{2017ApJ...836..200G}.  The \texttt{Starfish} framework builds a forward-model of the observed spectrum based spectrally emulated pre-computed synthetic model grid spectra.  We employed the PHOENIX synthetic model with an updated version of \texttt{Starfish} that maintains the native absolute fluxes from \citet{husser13}, rather than the inaccurate interpolation introduced in the Appendix of \citet{2017ApJ...836..200G}.

The likelihood-based Bayesian inference framework \texttt{Starfish} is for evaluating the posterior probability of parameters given an observed spectrum, \emph{assuming the model is correct}.  The model assumptions are:

\begin{itemize}
  \item Stellar photospheres are composed of exactly two distinct temperature components.
  \item Most of the spectral lines arise from known stellar atmospheric physics \citep{husser13}.
  \item The low temperature is above some finite threshold, typically $T>2400$ K.
  \item The two components share the same surface gravity and metallicity.
\end{itemize}

The inference framework provides a \emph{measurement} of the temperatures and projected filling factor of the two components.  The inference framework makes no effort to conduct \emph{model comparison}--- whether two temperature components are really needed over one, for example, or whether-or-not three components are warranted to capture additional, unexplained variance.  Nor does the framework conduct \emph{detection} of starspots-- detection has usually already been established by the high precision photometric time series observations that reveal conspicuous undulations indicative of distinct surface structures with distinct emergent photospheres.  We know the spots are there.  We simply aim to quantify the first moment of their physical properties-- the average coverage fractions, and the characteristic temperatures.



\section{Observations}\label{sec:obs}

\subsection{IGRINS Spectroscopy}\label{sec:igrins}

V827 Tau was observed on the Immersion Grating Infrared Spectrograph \citep[IGRINS]{park14} at the 2.7~m Harlan J. Smith Telescope at McDonald Observatory on 2014-11-21 at 05:23 UTC by observers SK Park, H Kim, JJ Lee.  The data were reduced with the Pipeline Package \citep{jaejoonlee15}.

\subsection{Long-term ground-based photometric monitoring} \label{sec:archival_phot}

We compiled many sources of publicly available $V-$band photometry datapoints including 236 from \citep{grankin08}, 552 from ASAS3 \citep{pojmanski04}, 473 from the AAVSO archive \citep{kafka16}, and 306 unpublished data points from the ASAS-SN survey \citep{shappee14}.  We spot-checked the diverse sources of photometry finding consistency in estimated flux levels when acquired near-contemporaneously.Figure \ref{fig:PhotTime} shows the compilation of all available photometry on V827 Tau.  Figure \ref{fig:PhotStamps} shows a zoom-in postage stamp on each season, phase folded with a period $P=3.75837$ days.

\begin{figure*}
 \centering
 \includegraphics[width=0.98\textwidth]{figures/V827_phot1990-2017.pdf}
 \caption{Compilation of archival and new flux measurements in $V-$band for V827 Tau, normalized to the global maximum flux value, which occurred in 1990.  The y-axis equals the minimum starspot area, when flux deficits are interpreted with the simplest possible analytic starspot model.}
 \label{fig:PhotTime}
\end{figure*}

\begin{figure*}
 \centering
 \includegraphics[width=0.98\textwidth]{figures/V827Tau_25season_stamps_V.pdf}
 \caption{Phase-folded $V-$band flux measurements of V827 Tau grouped into 25 observing seasons, and normalized to the global maximum in 1990.  The y-axis equals the minimum starspot area, when flux deficits are interpreted with the simplest possible analytic starspot model.}
 \label{fig:PhotStamps}
\end{figure*}

\subsection{K2 photometric monitoring} \label{sec:K2_obs}

We retrieved K2 Campaign 13 (C13) \cite{2014PASP..126..398H} lightcurves for V827 Tau, possessing EPIC ID 210698281.

%We retrieved 2MASS photometry \citep{skrutskie06}.

\section{Analysis}\label{sec:Analysis}
%Multiterm Lomb-Scargle periodograms \citep{ivezic14}, and Fourier series truncated to the first $\sim 4$ components]{vanderplas15a}.
 We performed a multiterm Lomb-Scargle periodogram \citep{ivezic14}, and Fourier series truncated to the first $\sim 4$ components \citep{vanderplas15a} on the K2 C13 lightcurve, yielding a 3.7584 day period.  We then fit a second order polynomial plus sine-and-cosine model with linear least squares.  The periodic minimum K2 flux corresponds to a flux decrement of 23.5\% from its maximum.  Figure \ref{fig:V827TauK2} shows the lightcurve overplotted with the sinusoidal model.

 \begin{figure*}
 \centering
 \includegraphics[width=0.98\textwidth]{figures/V827_Tau_K2C13_lightcurve.pdf}
 \caption{K2 C13 lightcurve of V827 Tau with a polynomial and sinusoidal model overplotted.  A few flares are perceptible.}
 \label{fig:V827TauK2}
\end{figure*}

We derived starspot temperatures and areal filling factors from IGRINS $H-$band spectra using the inference framework \texttt{Starfish} \citep{czekala15}, as extended by \citet{2017ApJ...836..200G}.  We ran 5000 samples with 40 \texttt{emcee} walkers \citep{foreman13}, with manual spot-checking of convergence.  We find typical filling factors around $f_\mathrm{spot} \sim 75\%$ and temperatures $T_\mathrm{spot} \sim 2600$ K.

We observed reasonably good agreedment when we overplotted our model prediction spectra with the data.  We do not expect the models to be perfect, due to known model mis-specifications such as imperfect line lists and un-modeled magnetic Zeeman broadening, to name a few.

The archival and K2 lightcurves constrain the minimum possible starspot coverage, assuming the Simplest Possible Analytic Starspot Model (SPASM).  Briefly, the SPASM interprets the lightcurve minimum as arising from a single, non-emitting, equatorial starspot on an edge-on star, and interprets the lightcurve maximum as \emph{spot-free}, that is, entirely devoid of starspots.  With these admittedly extreme assumptions, the flux minimum $F_{min}$ is exactly equal to the starspot areal surface coverage fraction, $f_{\mathrm{spot}}$.  The SPASM can be applied to a single season, such that there always exists a spot-free hemisphere, or can be applied to a global, long-term photometric dataset such as the one we have assembled for V827 Tau.  The global maximum flux for V827 Tau occurred on September 21, 1990, with $V=12.242$.Since then, the flux has faded 40\% to a median $V-$band magnitude of 12.63 in the 2016-2017 season.  The IGRINS spectrum was coincidentally acquired at the minimum of the rotationally modulated flux lightcurve, which corresponds to a global $F_{min} = f_{\mathrm{spot, min}} = 50\%$.

Figure \ref{} plots the constraints of starspot filling factor versus starspot temperature.  The burned-in MCMC draws from the posterior probability density distribution inferred from IGRINS spectra aggregated for all spectral orders are overplotted as a 2D-histogram.

 \begin{figure*}
 \centering
 \includegraphics[width=0.98\textwidth]{figures/V827_Tau_Starfish_results.pdf}
 \caption{Results of Starfish-based inference of starspot filling factor and temperature for 22 IGRINS $H-$band orders.  The size of the points is proportional to the strength of the posterior constraint on $f_{\rm spot}$ for each order.}
 \label{fig:V827TauStarfish}
\end{figure*}

\section{Conclusions}

Reiteration here.

\clearpage
\pagebreak


\appendix

\section{Are starspots confusing?}
\label{methods-details}

Short answer: no!

\acknowledgements

%ADS
We thank ADS!

%Kepler
This paper includes data collected by the Kepler mission. Funding for the Kepler mission is provided by the NASA Science Mission directorate.

% MAST
Some/all of the data presented in this paper were obtained from the Mikulski Archive for Space Telescopes (MAST). STScI is operated by the Association of Universities for Research in Astronomy, Inc., under NASA contract NAS5-26555.


{\it Facilities:} \facility{Smith (IGRINS)}, \facility{AAVSO}, \facility{CFHT (ESPaDOnS)}, \facility{INTEGRAL (OMC)}, \facility{ASAS}, \facility{CrAO:1.25m}, \facility{ARC (TripleSpec)}, \facility{Hale (DBSP)}, \facility{Gaia}

{\it Software: }
 \project{pandas} \citep{mckinney10},
 \project{emcee} \citep{foreman13},
 \project{matplotlib} \citep{hunter07},
 \project{numpy} \citep{vanderwalt11},
 \project{scipy} \citep{jones01},
 \project{ipython} \citep{perez07},
 \project{gatspy} \citep{JakeVanderplas2015},
 \project{starfish} \citep{czekala15},
 \project{seaborn} \citep{waskom14}
%\software{%
% \project{pandas} \citep{mckinney10}
%    \project{emcee} \citep{foreman13},
% \project{matplotlib} \citep{hunter07},
% \project{numpy} \citep{vanderwalt11},
% \project{scipy} \citep{jones01},
% \project{ipython} \citep{perez07},
% \project{gatspy} \citep{JakeVanderplas2015},
% \project{starfish} \citep{czekala15}}.

\clearpage

\bibliographystyle{apj}
\bibliography{ms}

\end{document}
